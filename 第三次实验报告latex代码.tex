\documentclass{article}
%导言区
\setlength{\parindent}{0pt}
\usepackage{listings}
\usepackage{ctex}
\usepackage{graphicx}
\usepackage[a4paper, body={18cm,22cm}]{geometry}
\usepackage{amsmath,amssymb,amstext,wasysym,enumerate,graphicx}
\usepackage{float,abstract,booktabs,indentfirst,amsmath}
\usepackage{array}
\usepackage{booktabs} %调整表格线与上下内容的间隔
\usepackage{multirow}
\usepackage{diagbox}
\usepackage[colorlinks,linkcolor=blue]{hyperref}
\graphicspath{{figures3/}}

%正文区
\begin{document}
\begin{titlepage}
\centering
\includegraphics[width=12cm,height=10cm]{logo}\\
\vspace{2cm}
{\Huge \heiti 系统开发工具基础实验报告\\} 

\vspace{4cm}
\begin{table}[h]
        \centering
        \begin{Large}
            \begin{tabular}{p{3cm} p{7cm}<{\centering}}
                上课时间: &  周五1-4节     \\ \cline{2-2}
                姓\qquad 名:      & 张仕达   \\ \cline{2-2}
                学\qquad 号: & 23010022094 \\ \cline{2-2}
                指导老师:       & 周小伟 \\ \cline{2-2}
            \end{tabular}
        \end{Large}     
    \end{table}
\end{titlepage}
\newpage % 插入新页  
\thispagestyle{empty} % 当前页不显示页码  
\section{实验内容}
\subsection{主题1:命令行环境:别名}  
\subsubsection{内容}
创建一个 dc 别名,它的功能是当我们错误的将 cd 输入为 dc 时也能正确执行。\\
在用户目录里建一个.bashrc的文件\\
加入:alias dc="cd"并保存即可\\
\includegraphics[width=0.8\textwidth]{test1_01}\\
\subsubsection{结果}  
\includegraphics[width=0.8\textwidth]{test1_02}\\
\vspace{1cm}
\subsection{主题2:命令行环境:远端设备}  
\subsubsection{内容}
为虚拟机配置一个密钥\\
\subsubsection{结果}  
\includegraphics[width=0.8\textwidth]{test2}\\
\vspace{1cm}
\subsection{主题3:python基础练习}  
\subsubsection{内容}
斐波那契数列\\
\includegraphics[width=0.8\textwidth]{test3_01}\\
\subsubsection{结果}  
\includegraphics[width=0.8\textwidth]{test3_02}\\
\vspace{1cm}
\subsection{主题4:python基础练习}  
\subsubsection{内容}
判断质数\\
\includegraphics[width=0.8\textwidth]{test4_01}\\
\subsubsection{结果}  
\includegraphics[width=0.8\textwidth]{test4_02}\\
\vspace{1cm}
\subsection{主题5:python基础练习}  
\subsubsection{内容}
返回一个数的阶乘\\
\includegraphics[width=0.8\textwidth]{test5_01}\\
\subsubsection{结果}  
\includegraphics[width=0.8\textwidth]{test5_02}\\
\vspace{1cm}
\subsection{主题6:python基础练习}  
\subsubsection{内容}
输出字符转长度\\
\includegraphics[width=0.8\textwidth]{test6_01}\\
\subsubsection{结果}  
\includegraphics[width=0.8\textwidth]{test6_02}\\
\vspace{1cm}
\subsection{主题7:python scipy}  
\subsubsection{内容}
打印角度单位,返回弧度\\
\includegraphics[width=0.8\textwidth]{test7_01}\\
\subsubsection{结果}  
\includegraphics[width=0.8\textwidth]{test7_02}\\
\newpage
\thispagestyle{empty}
\subsection{主题8:python scipy}  
\subsubsection{内容}
打印圆周率、黄金分割率\\
\includegraphics[width=0.8\textwidth]{test8_01}\\
\subsubsection{结果}  
\includegraphics[width=0.8\textwidth]{test8_02}\\
\vspace{1cm}
\subsection{主题9:python numpy}  
\subsubsection{内容}
np.arange返回一个向量组\\
\includegraphics[width=0.8\textwidth]{test9_01}\\
\subsubsection{结果}  
\includegraphics[width=0.8\textwidth]{test9_02}\\
\vspace{1cm}
\subsection{主题10:python numpy}  
\subsubsection{内容}
array: 向量;
arry.ndim: 向量的维度;\\
array.shape: 向量有几行几列\\
array.size: 向量的尺寸 2*3=6。\\
\includegraphics[width=0.8\textwidth]{test12_01}\\
\subsubsection{结果}  
\includegraphics[width=0.8\textwidth]{test12_02}\\
\subsection{主题11:python numpy}  
\subsubsection{内容}
向量的分割\\
\includegraphics[width=0.8\textwidth]{test10_01}\\
\subsubsection{结果}  
\includegraphics[width=0.8\textwidth]{test10_02}\\
\vspace{1cm}
\subsection{主题12:python numpy}  
\subsubsection{内容}
向量的合并\\
\includegraphics[width=0.8\textwidth]{test11_01}\\
\subsubsection{结果}  
\includegraphics[width=0.8\textwidth]{test11_02}\\
\vspace{1cm}
\subsection{主题13:python matplotlib}  
\subsubsection{内容}
绘制简单树状图\\
\includegraphics[width=0.8\textwidth]{test13_01}\\
\subsubsection{结果}  
\includegraphics[width=0.8\textwidth]{test13_02}\\
\vspace{1cm}
\subsection{主题14:python matplotlib}  
\subsubsection{内容}
绘制简单散点图\\
\includegraphics[width=0.8\textwidth]{test14_01}\\
\subsubsection{结果}  
\includegraphics[width=0.8\textwidth]{test14_02}\\
\newpage
\thispagestyle{empty}
\subsection{主题15:python matplotlib}  
\subsubsection{内容}
绘制简单饼状图\\
\includegraphics[width=0.8\textwidth]{test15_01}\\
\subsubsection{结果}  
\includegraphics[width=0.8\textwidth]{test15_02}\\
\vspace{1cm}
\subsection{主题16:python matplotlib}  
\subsubsection{内容}
绘制简单折线图\\
\includegraphics[width=0.8\textwidth]{test16_01}\\
\subsubsection{结果}  
\includegraphics[width=0.8\textwidth]{test16_02}\\
\vspace{1cm}
\subsection{主题17:python PIL}  
\subsubsection{内容}
1.调用图片\\
\includegraphics[width=0.8\textwidth]{test20_02}\\
2.可展示图片,也可保留图片,可保留不同格式,不同路径\\
\includegraphics[width=0.8\textwidth]{test20_01}\\
\subsubsection{结果}  
\includegraphics[width=0.8\textwidth]{test20_03}\\
\newpage
\thispagestyle{empty}
\subsection{主题18:python PIL}  
\subsubsection{内容}
实现镜像图片\\
\includegraphics[width=0.8\textwidth]{test18_01}\\
镜像前:\\
\includegraphics[width=0.8\textwidth]{test18_02}\\
镜像后:\\
\includegraphics[width=0.8\textwidth]{test18_03}\\
\subsubsection{结果}  
\vspace{1cm}
\subsection{主题19:python PIL}  
\subsubsection{内容}
1.设置两张图片
\includegraphics[width=0.8\textwidth]{test17_01}\\
2.设置新图片作拼接
\includegraphics[width=0.8\textwidth]{test17_02}\\
\subsubsection{结果}  
\includegraphics[width=0.8\textwidth]{test17_03}\\
\vspace{1cm}
\subsection{主题20:python PIL}  
\subsubsection{内容}
图片裁剪,去除多余白边\\
\includegraphics[width=0.8\textwidth]{test19_01}\\
\subsubsection{结果}  
裁剪前:\\
\includegraphics[width=0.8\textwidth]{test19_02}\\
裁剪后:\\
\includegraphics[width=0.8\textwidth]{test19_03}\\
\vspace{1cm}
\newpage
\thispagestyle{empty}
\section{解题感悟}  
这次的学习内容非常多,学的很累,第一次接触python后立马就跟着学了PIL,matplotlib等库,但总体来说收获也比前两次课要多的多。也是体会到了python这门语言的独特之处,希望可以借此机会好好深入了解一下python。\\
\href{https://github.com/coff-sug/system_development.git}{github地址[点击]}
\end{document}