\documentclass{article}
%导言区
\setlength{\parindent}{0pt}
\usepackage{float} % 引入float宏包
\usepackage{listings}
\usepackage{ctex}
\usepackage{graphicx}
\usepackage[a4paper, body={18cm,22cm}]{geometry}
\usepackage{amsmath,amssymb,amstext,wasysym,enumerate,graphicx}
\usepackage{float,abstract,booktabs,indentfirst,amsmath}
\usepackage{array}
\usepackage{booktabs} %调整表格线与上下内容的间隔
\usepackage{multirow}
\usepackage{diagbox}
\usepackage[colorlinks,linkcolor=blue]{hyperref}
\graphicspath{{figures/}}

%正文区
\begin{document}
\begin{titlepage}
\centering
\includegraphics[width=12cm,height=10cm]{logo}\\
\vspace{2cm}
{\Huge \heiti 系统开发工具基础实验报告\\} 

\vspace{4cm}
\begin{table}[h]
        \centering
        \begin{Large}
            \begin{tabular}{p{3cm} p{7cm}<{\centering}}
                上课时间: &  周五1-4节     \\ \cline{2-2}
                姓\qquad 名:      & 张仕达   \\ \cline{2-2}
                学\qquad 号: & 23010022094 \\ \cline{2-2}
                指导老师:       & 周小伟 \\ \cline{2-2}
            \end{tabular}
        \end{Large}     
    \end{table}
\end{titlepage}
\newpage % 插入新页  
\thispagestyle{empty} % 当前页不显示页码  
\section{实验内容}
\subsection{主题1:git bash 的基本配置:设置和查看用户的配置信息。}  
\subsubsection{内容}
1. 打开 Git Bash\\
2. 设置用户信息\\
git config –global user.name ”name”\\
git config –global user.email ”eamil”\\
3. 查看配置信息\\
git config –global user.name\\
git config –global user.email\\ 
\subsubsection{结果}  
\includegraphics[width=0.8\textwidth]{test1}\\
\vspace{1cm}
\subsection{主题2:git 的基本指令:创建本地仓库。}  
\subsubsection{内容}
1. 创建一个目录\\
2. 在目录内打开 Git Bash 窗口,执行命令 git init,将当前目录初始化为一个仓库\\
3. 创建成功后生成一个隐藏文件.git\\
\subsubsection{结果}   
\includegraphics[width=0.8\textwidth]{test3}\\
\subsection{主题3:git 的操作指令:提交文件至本地仓库。}  
\subsubsection{内容}
1. 将当前目录内的文件提交到暂存区 git add filename\\
2. 将所有文件都提交到暂存区 git add .\\
3. 将暂存区的文件提交到本店仓库\\
git commit -m ’ 注释内容’\\
\subsubsection{结果} 
\includegraphics[width=0.8\textwidth]{test4}\\ 
\subsection{主题4:git的操作指令:查看文件状态、查看提交日志}  
\subsubsection{内容}
1.每次提交文件到暂存区或本地仓库都可用 git status 查看文件状态\\
2.查看提交日志  git log [option]\\
其中[option]:
--all 显示所有分支\\
--pretty=oneline 将提交信息显示为一行\\
--abbrev-commit 使得输出的commitId更简短\\
--graph 以图的形式显示
\subsubsection{结果}  
\includegraphics[width=0.8\textwidth]{test5}\\
\subsection{主题5:git的操作指令:为常用指令配置别名}  
\subsubsection{内容}
1.打开用户目录,创建 .bashrc 文件,输入如下内容(举例):\\
用于输出git提交日志:\\
alias git-log='git log --pretty=oneline --all --graph --abbrev-commit'\\
用于输出当前目录所有文件及基本信息\\
alias ll='ls -al':\\
2."="右边''内的的指令可用左边指令代替,使用起来更加简便
\subsubsection{结果}  
\includegraphics[width=0.8\textwidth]{test6_01}\\
\includegraphics[width=0.8\textwidth]{test6_02}\\
\includegraphics[width=0.8\textwidth]{test6_03}\\
\vspace{1cm}
\subsection{主题6:git的操作指令:添加文件至忽略列表}  
\subsubsection{内容}
1.在当前目录中创建一个名为 .gitignore 的文件\\
2.在目录中输入相关信息可使相关文件不受git管理\\
例:输入 *.a  表示忽略目录中所有以.a结尾的文件
\subsubsection{结果}  
\includegraphics[width=0.8\textwidth]{test601}\\
\includegraphics[width=0.8\textwidth]{test602}\\
\vspace{1cm}
\subsection{主题7:git的操作指令:创建分支、切换分支}  
\subsubsection{内容}
1.创建分支 git branch 分支名\\
2.切换分支 git checkout 分支名\\
3.查看当前分支 git branch
\subsubsection{结果}  
\includegraphics[width=0.8\textwidth]{test7_01}\\
\includegraphics[width=0.8\textwidth]{test7_02}\\
\includegraphics[width=0.8\textwidth]{test7_03}\\
\newpage
\thispagestyle{empty}
\subsection{主题8:git的操作指令:解决合并冲突、合并分支、删除分支}  
\subsubsection{内容}
1.将一个分支上的提交合并到另一个分支 git merge 分支名\\
2.如果两个分支对文件的修改存在冲突则无法合并\\
3.需要对被合并的文件进行修改后再合并\\
4.git branch -d b1 删除分支\\
\subsubsection{结果}  
\includegraphics[width=0.8\textwidth]{test8_01}\\
\includegraphics[width=0.8\textwidth]{test8_02}\\
\vspace{1cm}
\subsection{主题9:git的操作指令:版本回退}  
\subsubsection{内容}
git reset --hard commitID\\
commitID 可以使用 git-log 或 git log 指令查看\\
\subsubsection{结果}  
\includegraphics[width=0.8\textwidth]{test9}\\
\vspace{1cm}
\subsection{主题10:git的操作指令:添加远程仓库}  
\subsubsection{内容}
1.初始化本地库之后,与已创建的远程仓库进行对接(示例为github)\\
命令: git remote add <远端名称> <仓库路径>\\
远端名称,默认是origin,取决于远端服务器设置\\
2.查看远端仓库 git remote
\subsubsection{结果}  
\includegraphics[width=0.8\textwidth]{test10}\\
\vspace{1cm}
\subsection{主题11:git的操作指令:将文件推送到远程仓库}  
\subsubsection{内容}
命令:git push  [远端名称] [本地分支名][:远端分支名]\\
如果远程分支名和本地分支名称相同,则可以只写本地分支\\
git push origin master\\
如果当前分支已经和远端分支关联,则可以省略分支名和远端名。\\
git push 将master分支推送到已关联的远端分支\\
\subsubsection{结果}  
\includegraphics[width=0.8\textwidth]{test11_01}\\
\includegraphics[width=0.8\textwidth]{test11_02}\\
\vspace{1cm}
\subsection{主题12:git的操作指令:从远程仓库克隆}  
\subsubsection{内容}
如果已经有一个远端仓库,我们可以直接clone到本地。\\
命令: git clone <仓库路径> [本地目录]\\
本地目录可以省略,会自动生成一个目录\\
\subsubsection{结果}  
\includegraphics[width=0.8\textwidth]{test12_01}\\
\includegraphics[width=0.8\textwidth]{test12_02}\\
\vspace{1cm}
\subsection{主题13:git的操作指令:从远程仓库抓取、拉取}  
\subsubsection{内容}
1.抓取 命令:git fetch [remote name] [branch name]\\
抓取指令就是将仓库里的更新都抓取到本地,不会进行合并\\
如果不指定远端名称和分支名,则抓取所有分支。\\
2.拉取 命令:git pull [remote name] [branch name]\\
拉取指令就是将远端仓库的修改拉到本地并自动进行合并,等同于fetch+merge\\
如果不指定远端名称和分支名,则抓取所有并更新当前分支\\
\subsubsection{结果}  
\includegraphics[width=0.8\textwidth]{test13_01}\\
\includegraphics[width=0.8\textwidth]{test13_02}\\
\includegraphics[width=0.8\textwidth]{test13_03}\\
\includegraphics[width=0.8\textwidth]{test13_04}\\
\vspace{1cm}
\subsection{主题14:latex对标题、作者、时间和引用的编辑}  
\subsubsection{内容}
在导言区插入标题、作者、时间的信息,在正文区进行引用\\
在标题之后加入引用\\
\subsubsection{结果}  
\includegraphics[width=0.8\textwidth]{test14_01}\\
\includegraphics[width=0.8\textwidth]{test14_02}\\
\newpage
\thispagestyle{empty}
\subsection{主题15:latex的标题架构}  
\subsubsection{内容}
标题设置:一级标题section{},耳机标题subsection{},三级标题subsubsection{};\\\\
段落设置:在一段的最后添加par代表一段的结束:\\\\
目录设置:在begin{document}内容中添加tableofcontents\\\\\\\\\\\\
\includegraphics[width=0.8\textwidth]{test15_01}
\subsubsection{结果} 
\includegraphics[width=0.8\textwidth]{test15_02}\\
\vspace{1cm}
\subsection{主题16:latex插入图片} 
\subsubsection{内容}
需要导入宏包:usepackage{graphicx}\\
导入图片所在的路径:graphicspath{{figures/}}%图片位于figures文件夹内
\\正文区引入图片:includegraphics[]{logo}%[]内设置图片参数 {}内输入图片名称
\subsubsection{结果}  
\includegraphics[width=0.8\textwidth]{test16_01}\\
\includegraphics[width=0.8\textwidth]{test16_02}\\
\vspace{1cm}
\subsection{主题17:latex插入表格}  
\subsubsection{内容}
\includegraphics[width=0.8\textwidth]{test17_01}\\
\subsubsection{结果}  
\includegraphics[width=0.8\textwidth]{test17_01}\\
\vspace{1cm}
\subsection{主题18:latex改变字体、大小}  
\subsubsection{内容}
使用代码:{字体 内容}\\
使用代码:{大小 内容}
\subsubsection{结果}  
\includegraphics[width=0.8\textwidth]{test18_01}\\
\includegraphics[width=0.8\textwidth]{test18_02}\\
\vspace{1cm}
\subsection{主题19:latex: 使用算法(伪代码)}  
\subsubsection{内容}
\includegraphics[width=0.8\textwidth]{test19_01}\\
\subsubsection{结果}  
\includegraphics[width=0.8\textwidth]{test19_02}\\
\vspace{1cm}
\subsection{主题20:latex:使用代码块}  
\subsubsection{内容}
\includegraphics[width=0.8\textwidth]{test20_01}\\
\subsubsection{结果} 
\includegraphics[width=0.8\textwidth]{test20_02}\\ 
\newpage
\thispagestyle{empty}
\section{解题感悟}  
 git和latex的学习对我来说是非常有益的。git让我实现了在本地仓库与远程仓库上对代码的管理,我可以使用git和其他人方便地共同开发一个项目,也可以将我的代码成果保存起来供其他人评价,更重要的是供我以后的学习回顾。而学习latex则教会了我对报告、论文模版的编辑,这对我以后进行数学建模等比赛有很大的裨益。\\\\
\href{https://github.com/coff-sug/system_development.git}{github地址[点击]}
\end{document}


