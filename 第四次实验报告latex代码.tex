\documentclass{article}
%导言区
\setlength{\parindent}{0pt}
\usepackage{listings}
\usepackage{ctex}
\usepackage{graphicx}
\usepackage[a4paper, body={18cm,22cm}]{geometry}
\usepackage{amsmath,amssymb,amstext,wasysym,enumerate,graphicx}
\usepackage{float,abstract,booktabs,indentfirst,amsmath}
\usepackage{array}
\usepackage{booktabs} %调整表格线与上下内容的间隔
\usepackage{multirow}
\usepackage{diagbox}
\usepackage[colorlinks,linkcolor=blue]{hyperref}
\graphicspath{{figures4/}}

%正文区
\begin{document}
\begin{titlepage}
\centering
\includegraphics[width=12cm,height=10cm]{logo}\\
\vspace{2cm}
{\Huge \heiti 系统开发工具基础实验报告\\} 

\vspace{4cm}
\begin{table}[h]
        \centering
        \begin{Large}
            \begin{tabular}{p{3cm} p{7cm}<{\centering}}
                上课时间: &  周五1-4节     \\ \cline{2-2}
                姓\qquad 名:      & 张仕达   \\ \cline{2-2}
                学\qquad 号: & 23010022094 \\ \cline{2-2}
                指导老师:       & 周小伟 \\ \cline{2-2}
            \end{tabular}
        \end{Large}     
    \end{table}
\end{titlepage}
\newpage % 插入新页  
\thispagestyle{empty} % 当前页不显示页码  
\section{实验内容}
\subsection{主题1:调试:获取最近一天中超级用户的登录信息及其所执行的指令}  
\subsubsection{内容}
使用 Linux 上的 journalctl 或 macOS 上的 log show 命令来获取最近一天中超级用户的登录信息及其所执行的指令。\\
如果找不到相关信息,您可以执行一些无害的命令,例如 sudo ls 然后再次查看。\\
\subsubsection{结果} 
\includegraphics[width=0.8\textwidth]{test01}\\
\vspace{1cm}
\subsection{主题2:修改键位映射}  
\subsubsection{内容}
利用PowerToys可修改键位映射\\
\subsubsection{结果}  
\includegraphics[width=0.8\textwidth]{test02}\\
\vspace{1cm}
\subsection{主题3:守护进程}  
\subsubsection{内容}
Linux中找出正在运行的所有守护进程\\
命令:systemctl status\\
\subsubsection{结果}  
\includegraphics[width=0.8\textwidth]{test03}\\
\vspace{1cm}
\subsection{主题4:常见命令行标志参数及模式}  
\subsubsection{内容}
一些操作的使用用法可以用\\
指令 --help 来查询\\
\subsubsection{结果}
\includegraphics[width=0.8\textwidth]{test04}\\  
\vspace{1cm}
\subsection{主题5:Markdown使用:设置标题}  
\subsubsection{内容}
使用*加标题名来设置\\
*数目的多少代表大小标题\\
\includegraphics[width=0.8\textwidth]{test05_01}\\  
\subsubsection{结果}  
\includegraphics[width=0.8\textwidth]{test05_02}\\  
\vspace{1cm}
\subsection{主题6:Markdown使用:粗体、斜体}  
\subsubsection{内容}
粗体:**粗体内容**\\
斜体:*斜体内容*\\
\includegraphics[width=0.8\textwidth]{test06_01}\\  
\subsubsection{结果}  
\includegraphics[width=0.8\textwidth]{test06_02}\\  
\vspace{1cm}
\subsection{主题7:Markdown使用:引用块、代码块}  
\subsubsection{内容}
引用\\
\includegraphics[width=0.8\textwidth]{test07_01}\\  
代码块\\
\includegraphics[width=0.8\textwidth]{test07_02}\\  
\subsubsection{结果}  
\includegraphics[width=0.8\textwidth]{test07_03}\\  
\newpage
\thispagestyle{empty}
\subsection{主题8:Markdown使用:列表}  
\subsubsection{内容}
有序列表\\
\includegraphics[width=0.8\textwidth]{test08_01}\\  
无序列表\\
\includegraphics[width=0.8\textwidth]{test08_02}\\
\subsubsection{结果}  
\includegraphics[width=0.8\textwidth]{test08_03}\\
\vspace{1cm}
\subsection{主题9:Markdown使用:表格}  
\subsubsection{内容}
\includegraphics[width=0.8\textwidth]{test09_01}\\  
\subsubsection{结果}  
\includegraphics[width=0.8\textwidth]{test09_02}\\  
\vspace{1cm}
\subsection{主题10:Markdown使用:数学公式}  
\subsubsection{内容}
\includegraphics[width=0.8\textwidth]{test010_01}\\  
\subsubsection{结果}  
\includegraphics[width=0.8\textwidth]{test010_02}\\  
\vspace{1cm}
\subsection{主题11:Markdown使用:Mermaid流程图}  
\subsubsection{内容}
\includegraphics[width=0.8\textwidth]{test11_01}\\  
\subsubsection{结果}  
\includegraphics[width=0.8\textwidth]{test11_02}\\  
\vspace{1cm}
\subsection{主题12:Markdown使用:Flowchart流程图}  
\subsubsection{内容}
\includegraphics[width=0.8\textwidth]{test12_01}\\  
\subsubsection{结果}  
\includegraphics[width=0.8\textwidth]{test12_02}\\  
\vspace{1cm}
\subsection{主题13:Markdown使用:时序图}  
\subsubsection{内容}
\includegraphics[width=0.8\textwidth]{test13_01}\\ 
\subsubsection{结果}  
\includegraphics[width=0.8\textwidth]{test13_02}\\ 
\vspace{1cm}
\subsection{主题14:Markdown使用:甘特图}  
\subsubsection{内容}
\includegraphics[width=0.8\textwidth]{test14_01}\\ 
\subsubsection{结果}  
\includegraphics[width=0.8\textwidth]{test14_02}\\ 
\newpage
\thispagestyle{empty}
\subsection{主题15:github的使用}  
\subsubsection{内容}
在github上创建仓库提交文件\\
\subsubsection{结果}  
\includegraphics[width=0.8\textwidth]{test15}\\ 
\vspace{1cm}
\subsection{主题16:pytorch入门}  
\subsubsection{内容}
torch.randperm(n) 函数用于生成一个从 0 到 n-1 的随机排列\\
\includegraphics[width=0.8\textwidth]{test16_01}\\ 
\subsubsection{结果}  
\includegraphics[width=0.8\textwidth]{test16_02}\\ 
\vspace{1cm}
\subsection{主题17:pytorch入门}  
\subsubsection{内容}
torch.eye(3)  生成单位矩阵\\
torch.zeros((2, 3))  生成零矩阵\\
torch.rand((2, 2)) 生成随机分布矩阵\\
\includegraphics[width=0.8\textwidth]{test17_01}\\
\subsubsection{结果}  
\includegraphics[width=0.8\textwidth]{test17_02}\\ 
\vspace{1cm}
\subsection{主题18:pytorch入门}  
\subsubsection{内容}
数学运算\\
\includegraphics[width=0.8\textwidth]{test18_01}\\
\subsubsection{结果}  
\includegraphics[width=0.8\textwidth]{test18_02}\\
\vspace{1cm}
\subsection{主题19:pytorch入门}  
\subsubsection{内容}
stack为拼接函数,函数的第一个参数为需要拼接的Tensor,第二个参数为细分到哪个维度\\
\includegraphics[width=0.8\textwidth]{test19_01}\\
\subsubsection{结果}  
\includegraphics[width=0.8\textwidth]{test19_02}\\
\vspace{1cm}
\subsection{主题20:pytorch入门}  
\subsubsection{内容}
使用自动微分机制配套使用SGD随机梯度下降来求最小值\\
\includegraphics[width=0.8\textwidth]{test20_01}\\
\subsubsection{结果}  
\includegraphics[width=0.8\textwidth]{test20_02}\\
\newpage
\thispagestyle{empty}
\section{解题感悟}  
这次实验内容很多,特别是大杂烩,而pytorch又是深度学习的内容,之前从来没有接触过,所以进度一直很慢。结束之后就发现收获还是很大的,为我以后的学习打下了一定的基础。希望以后能有更多的机会学习这种新知识。
\href{https://github.com/coff-sug/system_development.git}{github地址[点击]}
\end{document}