\documentclass{article}
%导言区
\setlength{\parindent}{0pt}
\usepackage{float} % 引入float宏包
\usepackage{listings}
\usepackage{ctex}
\usepackage{graphicx}
\usepackage[a4paper, body={18cm,22cm}]{geometry}
\usepackage{amsmath,amssymb,amstext,wasysym,enumerate,graphicx}
\usepackage{float,abstract,booktabs,indentfirst,amsmath}
\usepackage{array}
\usepackage{booktabs} %调整表格线与上下内容的间隔
\usepackage{multirow}
\usepackage{diagbox}
\usepackage[colorlinks,linkcolor=blue]{hyperref}
\graphicspath{{figures2/}}

%正文区
\begin{document}
\begin{titlepage}
\centering
\includegraphics[width=12cm,height=10cm]{logo.jpg}\\
\vspace{2cm}
{\Huge \heiti 系统开发工具基础实验报告\\} 

\vspace{4cm}
\begin{table}[h]
        \centering
        \begin{Large}
            \begin{tabular}{p{3cm} p{7cm}<{\centering}}
                上课时间: &  周五1-4节     \\ \cline{2-2}
                姓\qquad 名:      & 张仕达   \\ \cline{2-2}
                学\qquad 号: & 23010022094 \\ \cline{2-2}
                指导老师:       & 周小伟 \\ \cline{2-2}
            \end{tabular}
        \end{Large}     
    \end{table}
\end{titlepage}
\newpage % 插入新页  
\thispagestyle{empty} % 当前页不显示页码  
\section{实验内容}
\subsection{主题1:shell编程:自定义无参带返回值函数}  
\subsubsection{内容}
定义一个函数,求两个数的最大值\\
脚本内容:\\
\includegraphics[width=0.8\textwidth]{test01_01}\\
\subsubsection{结果} 
\includegraphics[width=0.8\textwidth]{test01_02}\\
\subsection{主题2:shell编程:自定义有参函数}  
\subsubsection{内容}
定义一个函数,返回指定参数\\
\includegraphics[width=0.8\textwidth]{test02_01}\\
\subsubsection{结果} 
\includegraphics[width=0.8\textwidth]{test02_02}\\
\subsection{主题3:shell工具:cut}  
\subsubsection{内容}
切割提取指定列数据\\
文本内容:\\
cu1.txt:\\
11 22 333\\
aa bb ccc\\
44 55 666\\
aa bb ccc\\
命令:按空格来分割切取第一列和第三列的内容\\
cut cu1.txt -d " " -f 1,3\\
\subsubsection{结果} 
\includegraphics[width=0.8\textwidth]{test03_01}\\
\subsection{主题4:shell工具:sed}  
\subsubsection{内容}
向指定行号前或后添加数据\\
文本内容:\\
cu1.txt:\\
11 22 333\\
aa bb ccc\\
44 55 666\\
aa bb ccc\\
命令:向第三行前添加qqq"\\
sed "3iqqq" cu1.txt\\
3表示行号,i表示向前添加\\
命令:向第三行后添加"qqq"\\
sed "3aqqq" cu1.txt\\
3表示行号,a表示向后添加\\
\subsubsection{结果} 
\includegraphics[width=0.8\textwidth]{test03_02}\\\\
\includegraphics[width=0.8\textwidth]{test03_03}\\
\subsection{主题5:shell工具:awk}  
\subsubsection{内容}
awk是一种文本分析工具,默认按照每行空格切割数据\\
命令:\\
\includegraphics[width=0.8\textwidth]{test04_01}\\
\subsubsection{结果} 
\includegraphics[width=0.8\textwidth]{test04_02}\\
\subsection{主题6:vim: vimrc的配置}  
\subsubsection{内容}
将.vimrc文件放在用户的目录下面,就会对vim进行配置,配置后vim的功能将更加完善。\\
\includegraphics[width=0.8\textwidth]{test05_01}\\
\subsubsection{结果} 
\includegraphics[width=0.8\textwidth]{test05_02}\\
\subsection{主题7:数据处理}  
\subsubsection{内容}
查询文件中含有"aa"的行数据\\
文本内容:\\
cu1.txt:\\
11 22 333\\
aa bb ccc\\
44 55 666\\
aa bb ccc\\
命令:\\
sed -n '/aa/p' cu1.txt
\subsubsection{结果} 
\includegraphics[width=0.8\textwidth]{test06_01}\\
\subsection{主题8:数据处理:查询一个文件(file.txt)中空行所在的行号}  
\subsubsection{内容}
脚本实现\\
1.file.txt 数据准备\\
test test\\
\\
111 111\\
222 222\\
\\
333 333\\
2.脚本实现\\
1.检查file.txt文件是否存在  \\
2.使用grep查找空行,并打印行号  \\
\includegraphics[width=0.8\textwidth]{test1_01}\\
\subsubsection{结果}  
\includegraphics[width=0.8\textwidth]{test1_02}\\
\vspace{1cm}
\subsection{主题9:数据处理:对一个文件中的每一行第一个数字进行排序}  
\subsubsection{内容}
1.file.txt 数据准备\\
\includegraphics[width=0.8\textwidth]{test2_01}\\
2.脚本实现\\
检查file.txt文件是否存在 \\
使用awk提取数字行,并通过sort命令进行排序,最后输出结果\\
假设每行都是数字或者包含数字(但只关注第一个数字)\\
\includegraphics[width=0.8\textwidth]{test2_02}\\
\subsubsection{结果}   
\includegraphics[width=0.8\textwidth]{test2_03}\\
\subsection{主题10:数据处理:查找root下所有包含“111”的文件名称}  
\subsubsection{内容}
脚本实现\\
1.使用find命令来遍历根目录\\
2.使用chmod +x 脚本名 命令给予该脚本执行权限。\\
\includegraphics[width=0.8\textwidth]{test3_01}\\
\subsubsection{结果}  
\includegraphics[width=0.8\textwidth]{test3_02}\\
\subsection{主题11:数据处理:输出文件内长度大于3的单词}  
\subsubsection{内容}
1.数据准备\\
aaar aaas asd\\
r\\
aaf as\\
2.脚本实现\\
\includegraphics[width=0.8\textwidth]{test11_01}\\
\subsubsection{结果}  
\includegraphics[width=0.8\textwidth]{test11_02}\\
\subsection{主题12:数据处理:单词去重排序}  
\subsubsection{内容}
文件内容:\\
\includegraphics[width=0.8\textwidth]{test12_01}\\
脚本内容:\\
\includegraphics[width=0.8\textwidth]{test12_02}\\
\subsubsection{结果}  
\includegraphics[width=0.8\textwidth]{test12_03}\\
\vspace{1cm}
\subsection{主题13:shell脚本编程:批量生成文件}  
\subsubsection{内容}
\includegraphics[width=0.8\textwidth]{test4_01}\\
\subsubsection{结果}  
\includegraphics[width=0.8\textwidth]{test4_02}\\
\vspace{1cm}
\subsection{主题14:shell脚本编程:猜数字游戏}  
\subsubsection{内容}
脚本内容:\\
\includegraphics[width=0.8\textwidth]{test13_01}\\
\subsubsection{结果}  
\includegraphics[width=0.8\textwidth]{test13_02}\\
\newpage
\thispagestyle{empty}
\subsection{主题15:shell脚本编程:输入三个数并进行升序排序}  
\subsubsection{内容}
\includegraphics[width=0.8\textwidth]{test14_01}\\
\subsubsection{结果}  
\includegraphics[width=0.8\textwidth]{test14_02}\\
\vspace{1cm}
\subsection{主题16:shell脚本编程:石头、剪刀、布游戏} 
\subsubsection{内容}
\includegraphics[width=0.8\textwidth]{test16_01}\\
\subsubsection{结果}  
\includegraphics[width=0.8\textwidth]{test16_02}\\
\vspace{1cm}
\subsection{主题17:shell脚本编程:打印乘法口诀表}  
\subsubsection{内容}
\includegraphics[width=0.8\textwidth]{test17_01}\\
\subsubsection{结果}  
\includegraphics[width=0.8\textwidth]{test17_02}\\
\vspace{1cm}
\subsection{主题18:shell脚本编程:对1到100进行求和}  
\subsubsection{内容}
\includegraphics[width=0.8\textwidth]{test18_01}\\
\subsubsection{结果}  
\includegraphics[width=0.8\textwidth]{test18_02}\\
\vspace{1cm}
\subsection{主题19:shell脚本编程:打印国际象棋棋盘}  
\subsubsection{内容}
\includegraphics[width=0.8\textwidth]{test19_01}\\
\subsubsection{结果}  
\includegraphics[width=0.8\textwidth]{test19_02}\\
\vspace{1cm}
\subsection{主题20:判断用户输入字符类型}  
\subsubsection{内容}
\includegraphics[width=0.8\textwidth]{test20_01}\\
\subsubsection{结果} 
\includegraphics[width=0.8\textwidth]{test20_02}\\ 
\newpage
\thispagestyle{empty}
\section{解题感悟}  
  shell编程的学习让我对文件的操作更加熟练了,再加上vim的使用,我现在能够更加方便的对各种文件进行修改。类似编程、功能却更加偏实用的脚本则让我对于代码的兴趣更加的浓厚。这次课程的学习让我受益匪浅。\\\\
\href{https://github.com/coff-sug/system_development.git}{github地址[点击]}
\end{document}
